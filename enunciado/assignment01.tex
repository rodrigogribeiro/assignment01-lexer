% Created 2023-10-28 sáb 10:04
% Intended LaTeX compiler: pdflatex
\documentclass[a4paper,11pt]{article}
\usepackage[utf8]{inputenc}
\usepackage[T1]{fontenc}
\usepackage{graphicx}
\usepackage{longtable}
\usepackage{wrapfig}
\usepackage{rotating}
\usepackage[normalem]{ulem}
\usepackage{amsmath}
\usepackage{amssymb}
\usepackage{capt-of}
\usepackage{hyperref}
\usepackage[table]{xcolor}
\usepackage[margin=0.9in,bmargin=1.0in,tmargin=1.0in]{geometry}
\usepackage{algorithm2e}
\usepackage{algorithm}
\usepackage{amsmath}
\usepackage{arydshln}
\usepackage{subcaption}
\newcommand{\point}[1]{\noindent \textbf{#1}}
\usepackage{hyperref}
\usepackage{csquotes}
\usepackage{graphicx}
\usepackage{bm}
\usepackage{subfig}
\usepackage[mla]{ellipsis}
\parindent = 0em
\setlength\parskip{.5\baselineskip}
\usepackage{pgf}
\usepackage{tikz}
\usetikzlibrary{shapes,arrows,automata,quotes}
\usepackage[latin1]{inputenc}
\usepackage{adjustbox,listings}
\author{Prof. Rodrigo Ribeiro}
\date{28-10-2023}
\title{Trabalho prático 1 - Análise léxica.\\\medskip
\large Construção de Compiladores I}
\hypersetup{
 pdfauthor={Prof. Rodrigo Ribeiro},
 pdftitle={Trabalho prático 1 - Análise léxica.},
 pdfkeywords={latex, org-mode, writing},
 pdfsubject={my org-mode to latex templates},
 pdfcreator={Emacs 28.2 (Org mode 9.7)}, 
 pdflang={English}}
\begin{document}

\maketitle
\section*{1. Instruções importantes}
\label{sec:orgbb564ce}

Nesta seção são apresentadas diversas informações relevantes
referentes a entrega do trabalho e orientações a serem seguidas
durante a implementação do mesmo.

Leia atentamente antes de começá-lo.
\subsection*{1.1. Equipe de desenvolvimento}
\label{sec:orga686ec3}

O trabalho será desenvolvido por grupos de até dois alunos.
Não será permitido o trabalho ser realizado por três ou
mais pessoas.
\subsection*{1.2. Escolha da linguagem de programação}
\label{sec:orgb81946b}

O trabalho deverá ser desenvolvido na linguagem Haskell.
No que diz respeito a implementação, todas as estruturas de
dados referente a modelagem e desenvolvimento do trabalho
deverão ser implementadas. Poderão ser usadas apenas
bibliotecas de estruturas de dados elementares (listas,
filas, pilhas, árvores, etc.), leitura de strings e arquivos ou
ferramentas de uso geral, como o gerador de analisadores
léxicos \href{https://haskell-alex.readthedocs.io/en/latest/}{Alex}.
\subsection*{1.3. Artefatos a serem entregues}
\label{sec:orge43983f}

Os artefatos a serem entregues são:

\begin{itemize}
\item Código fonte do programa;
\item Relatório do trabalho em formato pdf.
\end{itemize}

Antes de enviar seu trabalho para avaliação,
assegure-se que:

\begin{itemize}
\item Seu código compila e executa em ambiente Unix / Linux.
Programas que não compilam receberão nota zero;
\item Todos os fontes a serem enviados têm, em comentário no
início do arquivo, nome e matrícula do(s) autor(es) do
trabalho;
\item Arquivo do relatório tenha a identificação do(s)
autor(es) do trabalho;
\end{itemize}
\subsection*{1.4. Critérios de avaliação}
\label{sec:orgb8b699b}

A avaliação será feita mediante análise do código fonte,
relatório e estrutura de commits do repositório.
Os seguintes fatores serão observados na avaliação do código
fonte: corretude do programa, estrutura do código
e legibilidade. A corretude se refere à implementação correta
de todas as funcionalidades especificadas, i.e., se o programa
desenvolvido está funcionando corretamente e não apresenta erros.
Os demais fatores avaliados no código fonte são referentes a
organização e escrita do trabalho.

Nesse quesito, será observado:

\begin{itemize}
\item Estruturas de dados bem planejadas;

\item Código modularizado em nível físico (separação em arquivos)
e lógico (arquitetura bem definida);

\item Legibilidade do código, i.e., nomes adequados de variáveis,
funções e comentários úteis no código;
\end{itemize}

O relatório do código deve conter informações relevantes para
compilar, executar e auxiliar no entendimento da estratégia adotada
para resolução do problema e do código fonte.

Ressalta-se que no relatório não deve conter cópias do fonte --
afinal o seu fonte é um dos artefatos entregue --, porém trechos de
códigos podem ser incluídos caso isso facilite a explicação e
elucidação das estratégias utilizadas.

O relatório deve apresentar as decisões de projetos tomadas:
expressões regulares e autômatos para os tokens da linguagem,
estruturas de dados usadas, estratégia de implementação do analisador
léxico e detalhes de leitura da entrada, dentre outras informações.

Tenha atenção ao prazo de entrega, pois não serão permitidas
alterações nos repositórios após a data limite.
\section*{2. Especificação técnica do trabalho}
\label{sec:org7a67e82}

Ao longo do semestre será construído um compilador para a linguagem lang.
Assim, neste primeiro trabalho pede-se que seja implementado o analisador léxico
desta linguagem.

Deve-se implementar um programa de teste que usa o analisador léxico implementado.
Tal programa irá receber como entrada um arquivo contendo um programa na linguagem e
deverá imprimir, na saída padrão, a sequência de tokens produzido pelo
analisador léxico, um token por linha.

Por exemplo, para o programa da Figura 1 a saída será:
\begin{lstlisting}
ID:main
(
)
{
PRINT
ID:fat
(
INT:10
...
\end{lstlisting}


\begin{figure}
\begin{lstlisting}[language=uL]
main() {
  print fat(10)[0];
}

fat(num :: Int) : Int {
  if (num < 1)
    return 1;
  else
    return num * fat(num-1)[0];
}

divmod(num :: Int, div :: Int) : Int, Int {
  q = num / div;
  r = num % div;
  return q, r;
}

\end{lstlisting}
\caption{Exemplos de funções e procedimentos  na linguagem lang.}
\label{fig:prog}
\end{figure}
Espera-se que o programa produzido forneça uma interface de linha de comandos
que receba o arquivo de entrada e produza mensagens de erro / ajuda de forma
adequada.
\section*{3. Entrega do Trabalho}
\label{sec:org260cf3c}

A data da entrega do trabalho será até o dia \textbf{12 de novembro de 2023}.
\end{document}